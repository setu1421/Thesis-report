During the past few years, many works on various
multi-label image classification models have been conducted. These models are generally based on two
types of frameworks: bag-of-words (BoW) \cite{18}, \cite{12},
\cite{13}, \cite{14} and deep learning \cite{19}.

\section{Bag-of-Words Based Models}

A traditional BoW model is composed of multiple
modules, e.g., feature representation, classification
and context modeling. For feature representation,
the main components include hand-crafted feature
extraction, feature coding and feature pooling, which
generate global representations for images. Specifically, hand-crafted features, such as SIFT \cite{3}, Histogram of Oriented Gradients \cite{20} and Local Binary
Patterns \cite{21} are firstly extracted on dense grids or
sparse interest points and then quantized by different
coding schemes, e.g., Vector Quantization \cite{22}, Sparse
Coding and Gaussian Mixture Models \cite{23}. These
encoded features are finally pooled by feature aggregation methods, such as Spatial Pyramid Matching
(SPM) \cite{4}, to form the image-level representation. For
classification, conventional models, such as SVM \cite{6}
and random forests \cite{7}, are utilized. Beyond conventional modeling methods, many recent works \cite{18}, \cite{13} have demonstrated that the usage
of context information, e.g., spatial location of object
and background scene from the global view, can
considerably improve the performance of multi-label
classification and object detection. \hfill \break

Although these works have made great progress
in visual recognition tasks, the involved hand-crafted
features are not always optimal for particular tasks.
Recently, in contrast to hand-crafted features, learnt
features with deep learning structures have shown
great potential for various vision recognition tasks,
which will be introduced in the following subsection.

\section{Deep Learning Based Models}

Deep learning tries to model the high-level abstractions of visual data by using architectures composed of multiple non-linear transformations. Specifically, deep convolutional neural network (CNN) \cite{8}
has demonstrated an extraordinary ability for image classification \cite{9}, \cite{10} on single-label datasets such as CIFAR-10/100 \cite{17} and ImageNet \cite{2}.
More recently, CNN architectures have been
adopted to address multi-label problems. Gong et
al. \cite{19} studied and compared several multi-label
loss functions for the multi-label annotation problem
based on a similar network structure to \cite{10}. However,
due to the large number of parameters to be learned
for CNN, an effective model requires lots of training
samples. Therefore, training a task-specific convolutional neural network is not applicable on datasets
with limited numbers of training samples.







