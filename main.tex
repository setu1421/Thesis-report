% ###########################################################
%
% LaTeX Template for undergraduate thesis book and report
%
% Author: Md. Ashiqur Rahman (Opu)
% Roll: 1007035
% 
% Special Contents:
%   * Sub-figures
%   * Tikz Image for flowchart
%
% Rename the chapters as per your requirements
%
% Please do not remove Template author name from this file
%
% ###########################################################


\documentclass[12pt]{report}
\usepackage[utf8x]{inputenc}
\usepackage{graphicx}
\graphicspath{{images/}}
\usepackage{hyperref}
%\usepackage{chngpage}
%\usepackage{subfig}
%\usepackage{subfigure}
%\usepackage{float}
\usepackage[a4paper,tmargin=1.1in,bmargin=1.1in,lmargin=1.2in,rmargin=1in]{geometry}
%\usepackage{fancyhdr}
%\pagestyle{fancy}
\usepackage{enumitem}
\usepackage[algoruled,boxed,lined]{algorithm2e}
\usepackage{graphicx}
\usepackage{caption}
\usepackage[font={scriptsize}]{subcaption}
%\usepackage{subcaption}
\usepackage{array}
\usepackage{tikz}
\usetikzlibrary{shapes.geometric,arrows,positioning}
\tikzstyle{startstop}=[ellipse, minimum width=3cm, minimum height=0.6cm,text centered, draw=black, fill=white]
\tikzstyle{io}=[trapezium, trapezium left angle=70, trapezium right angle=110, minimum width=3cm, minimum height=0.6cm, text centered, draw=black, fill=white]
\tikzstyle{process}=[rectangle, minimum width=2cm, minimum height=0.6cm, text centered, draw=black, fill=white]
\tikzstyle{decision}=[diamond, aspect=2, minimum width=1.5cm, minimum height=0.8cm, text centered, draw=black, fill=white]
\tikzstyle{arrow} = [thick,->,>=stealth]
\tikzset{draw}

\usepackage{amsthm}
\usepackage{amsmath}
\usepackage{amssymb}

\theoremstyle{plain}
\newtheorem{thm}{Theorem}[chapter] % reset theorem numbering for each chapter
\newtheorem{theory}{Theorem}
\newtheorem{tlem}{Theorem}
\newtheorem{texmpl}{Theorem}
\theoremstyle{definition}
\newtheorem{defn}[thm]{Definition} % definition numbers are dependent on theorem numbers
\newtheorem{exmp}[texmpl]{Example} % same for example numbers
\newtheorem{lem}[tlem]{Lemma}

\newcommand\textbox[1]{%
  \parbox{.333\textwidth}{#1}%
}


\usepackage{titlesec}
 

\makeatletter
\def\@makechapterhead#1{%
  \vspace*{0\p@}%
  {\parindent \z@ \centering
    \normalfont
    \ifnum \c@secnumdepth >\m@ne
      \if@mainmatter
        \large \bfseries \@chapapp\space \thechapter
        \par\nobreak
        \vskip 20\p@
      \fi
    \fi
    \interlinepenalty\@M
    \LARGE \bfseries #1\par\nobreak
    \vskip 40\p@
  }}
\def\@schapter#1{\if@twocolumn
                   \@topnewpage[\@makeschapterhead{#1}]%
                 \else
                   \@makeschapterhead{#1}%
                   \@afterheading
                 \fi}
\def\@makeschapterhead#1{%
  \vspace*{0\p@}%
  {\parindent \z@ \centering
    \normalfont
    \interlinepenalty\@M
    \Large \bfseries  #1\par\nobreak
    \vskip 40\p@
  }}

\renewcommand\section{\@startsection {section}{1}{\z@}%
                                   {-6ex \@plus -1ex \@minus -.2ex}%
                                   {2.3ex \@plus.2ex}%
                                   {\normalfont\Large\bfseries}}

\renewcommand\subsection{\@startsection {subsection}{1}{\z@}%
                                   {-4ex \@plus -1ex \@minus -.2ex}%
                                   {2.3ex \@plus.2ex}%
                                   {\normalfont\large\bfseries}}

\makeatother

\begin{document}

\pagenumbering{roman}
\setcounter{page}{1}


% ############################################################################################################
%
% Title Page (only if you want to make it with LaTeX)
%
\begin{figure}
  \centering
  \includegraphics[width=2cm]{kuetlogo.png}
\end{figure}

\title{
    {An Approach for Multi Label Image Classification Using Single Label Convolutional Neural Network Classifier with Objectness Measure and Selective Search}\\
    {\includegraphics[width=4cm]{gap.png}}\\
    {\includegraphics[width=4cm]{gap.png}}\\
    {\includegraphics[width=4cm]{gap.png}}\\
}

\author{Setu Basak\\
        Roll: 1107043\\
        Shubhashis Karmakar\\
        Roll: 1107001\\\\
        Supervisor: Dr. Kazi Md. Rokibul Alam\\
        Professor, CSE, KUET\\\\\\
}

\date{April, 2016}
%
% ############################################################################################################



\setlength{\parindent}{1em}
%\setlength{\parskip}{1em}

%\usepackage{titlesec}

\titlespacing*{\section}
{0pt}{5.5ex plus 1ex minus .2ex}{4.3ex plus .2ex}
\titlespacing*{\subsection}
{0pt}{5.5ex plus 1ex minus .2ex}{4.3ex plus .2ex}

%\begin{document}

\maketitle

\addtocontents{toc}{~\hfill\textbf{Page}\par}

\chapter*{Acknowledgement}
\addcontentsline{toc}{chapter}{Acknowledgement}
%\renewcommand{\abstractname}{Acknowledgements}
%\begin{abstract}
With due homage and honor, we want to express our gratitude to Almighty Allah.\\


\noindent We express our indebtedness to our dedicated supervisor Dr. Kazi Md. Rokibul Alam (Professor, Department of Computer Science and Engineering, KUET) for his necessary guidance, suggestions and encouragement to all phases of our work. We
would also like to thank all the teachers who've been much helpful and encouraging in completion of our thesis work.\\

\noindent\textbox{\hfill}\textbox{\hfil \hfil}\textbox{\hfill \textbf{Authors}}
%\end{abstract}

\chapter*{Abstract}
\addcontentsline{toc}{chapter}{Abstract}
%\renewcommand{\abstractname}{Abstract}
%\begin{abstract}
Single label image classification has been promisingly demonstrated using Convolutional Neural Network (CNN). However, how this CNN will fit for multi-label images is still difficult to solve. It is mainly difficult due to the complex underlying
object layouts and insufficient multi-label training images. In this work, we propose an approach for classifying multi-label image by a trained single label classifier using CNN with objectness measure and selective search. We took two established image segmentation techniques for segmenting a single image which is a multi-label image into some segmented images. Then we forwarded the images to our trained CNN and predicted the labels of the segmented images by generalizing the result. Our single-label image classifier gives 87\% accuracy on CIFAR-10 dataset. Using objectness measure with CNN gives us 51\% accuracy on a multi-label dataset and gives upto 57\% accuracy using selective search both considering top-4 labels which is significantly good for a simple approach rather than a complex approach of multi-label classifier using CNN.

%\end{abstract}


%\chapter*{Dedication}
%To mum and dad

\tableofcontents
%\addcontentsline{toc}{chapter}{Contents}
\listoffigures
\addcontentsline{toc}{chapter}{List of Figures}
%\listoftables
%\addcontentsline{toc}{chapter}{List of Tables}

%\setlength{\parskip}{1ex plus 0.5ex minus 0.2ex}

\chapter{Introduction}
\setcounter{page}{2}
\pagenumbering{arabic}
\input{chapters/introduction}

\chapter{Related Works}
\input{chapters/literature}

\chapter{Initial Attempts}
\input{chapters/prevapproaches}

\chapter{Dataset}
\input{chapters/dataset}

\chapter{The Architecture}
\input{chapters/systemmodel}

%\chapter{Proposed Algorithm}
%\input{chapters/proposedalgo}

\chapter{Experimental Result}
\input{chapters/performance}

\chapter{Tools and Recommendations for Future Works}
\input{chapters/tools}

\chapter{Conclusion}
\input{chapters/conclusion}

%\appendix
%\chapter{Appendix}
%\input{chapters/appendix}

\begin{thebibliography}{150}

    % Set Reference ID carefully, do not use replication
    \bibitem {1} L. Fei-Fei, R. Fergus, and P. Perona. Learning generative visual
models from few training examples: An incremental bayesian
approach tested on 101 object categories. Computer Vision and
Image Understanding, 106(1):59–70, 2007.

\bibitem {2} J. Deng, W. Dong, R. Socher, L.-J. Li, K. Li, and L. FeiFei. Imagenet: A large-scale hierarchical image database. In
Computer Vision and Pattern Recognition, pages 248–255, 2009.

\bibitem {3} D. G. Lowe. Distinctive image features from scale-invariant
keypoints. International Journal of Computer Vision, 60(2):91–
110, 2004.

\bibitem {4} S. Lazebnik, C. Schmid, and J. Ponce. Beyond bags of features:
Spatial pyramid matching for recognizing natural scene categories. In Computer Vision and Pattern Recognition, volume 2,
pages 2169–2178, 2006.

\bibitem {5} J. Wang, J. Yang, K. Yu, F. Lv, T. Huang, and Y. Gong. Localityconstrained linear coding for image classification. In Computer
Vision and Pattern Recognition, pages 3360–3367, 2010.

\bibitem {6} C.-C. Chang and C.-J. Lin. Libsvm: a library for support
vector machines. ACM Trans. Intelligent Systems and Technology,
2(3):27, 2011.

\bibitem {7}  L. Breiman. Random forests. Machine learning, 45(1):5–32, 2001.

\bibitem {8} Y. LeCun, B. Boser, J. Denker, D. Henderson, R. Howard,
W. Hubbard, and L. Jackel. Handwritten digit recognition with
a back-propagation network. In Neural Information Processing
Systems, 1990.

\bibitem {9} K. Jarrett, K. Kavukcuoglu, M. Ranzato, and Y. LeCun. What
is the best multi-stage architecture for object recognition? In
International Conference on Computer Vision, pages 2146–2153,
2009.

\bibitem {10} A. Krizhevsky, I. Sutskever, and G. Hinton. Imagenet classification with deep convolutional neural networks. In Neural
Information Processing Systems, pages 1106–1114, 2012.

\bibitem {11} W. Ouyang and X. Wang. Joint deep learning for pedestrian
detection. In International Conference on Computer Vision, pages
2056–2063, 2013.

\bibitem {12} F. Perronnin, J. Sanchez, and T. Mensink. Improving the ´
fisher kernel for large-scale image classification. In European
Conference on Computer Vision, pages 143–156, 2010.

\bibitem {13} Q. Chen, Z. Song, Y. Hua, Z. Huang, and S. Yan. Hierarchical
matching with side information for image classification. In
Computer Vision and Pattern Recognition, pages 3426–3433, 2012.

\bibitem {14} J. Dong, W. Xia, Q. Chen, J. Feng, Z. Huang, and S. Yan.
Subcategory-aware object classification. In Computer Vision and
Pattern Recognition, pages 827–834, 2013.

\bibitem {15} Alexe, B., Deselares, T. and Ferrari.
 Measuring the objectness of image windows. V. PAMI 2012.

\bibitem {16} R. R. Uijlings, Koen E. A. van de Sande, Theo Gevers, Arnold W. M. Smeulders. 
Selective Search for Object Recognition, Jasper  International Journal of Computer Vision, Volume 104 (2), page 154-171, 2013.

\bibitem {17} Dataset of CIFAR-10 https://www.cs.toronto.edu/~kriz/cifar.html.

\bibitem {18} H. Harzallah, F. Jurie, and C. Schmid. Combining efficient
object localization and image classification. In Computer Vision
and Pattern Recognition, pages 237–244, 2009.

\bibitem {19} Y. Gong, Y. Jia, T. K. leung, A. Toshev, and S. Ioffe. deep
convolutional ranking for multi label image annotation. In
International Conference on Learning Representations, 2014.

\bibitem {20} N. Dalal and B. Triggs. Histograms of oriented gradients for
human detection. In Computer Vision and Pattern Recognition,
volume 1, pages 886–893, 2005.

\bibitem {21} T. Ojala, M. Pietikainen, and D. Harwood. A comparative 
study of texture measures with classification based on featured
distributions. Pattern recognition, 29(1):51–59, 1996.

\bibitem {22} N. M. Nasrabadi and R. A. King. Image coding using vector quantization: A review. IEEE Trans. Communications, 36(8):957–
971, 1988.

\bibitem {23} P. Hedelin and J. Skoglund. Vector quantization based on
gaussian mixture models. IEEE Trans. Speech and Audio Processing, 8(4):385–401, 2000.

\bibitem{24} D. Comaniciu and P. Meer. Mean shift: a robust approach
toward feature space analysis. TPAMI, 24:603–619, 2002.

\bibitem{25} P. F. Felzenszwalb and D. P. Huttenlocher. Efficient GraphBased Image Segmentation. IJCV, 59:167–181, 2004. 

\bibitem{26} Sergey Ioffe, Christian Szegedy.
 Batch Normalization: Accelerating Deep Network Training by Reducing
Internal Covariate Shift.

\end{thebibliography}


\end{document}
